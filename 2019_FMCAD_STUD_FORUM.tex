% Oliver Kullmann, 18.5.2019 (Swansea)
% Bibtex entry \cite{KullmannShukla2019DQCNF}.
% Underlying report: 2018_DQBF.tex \cite{KullmannShukla2018DQBF}.
% Submitted to FMCAD 2019 https://fmcad.forsyte.at/FMCAD19/ :
% ID 79d524970076dd8509ffbd92da4922bba7682e3b.

\documentclass[conference]{IEEEtran}
\IEEEoverridecommandlockouts


\input Latex_macros/Definitionen.tex

\usepackage{enumerate}
\usepackage[all]{xy}
\usepackage[active]{srcltx}

\interdisplaylinepenalty=2500

\DeclareMathOperator{\varess}{\var_{es}} % essential variables

\DeclareMathOperator{\Aaut}{A}
\DeclareMathOperator{\Eaut}{E}


\begin{document}

\title{Improving Reasoning on DQBF}


\author{
\IEEEauthorblockN{Ankit Shukla}
\IEEEauthorblockA{
Johannes Kepler University, Austria\\
Email: ankit.shukla@jku.at}
}

\maketitle

\begin{abstract}
In this paper we investigate two techniques, autarkies and symmetry to improve the reasoning with the Dependency Quantified Boolean Formulas (DQBF).
%
DQBF extend QBF with non-linear dependencies between the quantified variables.
%
Inspired by the fruitfulness of the established model for generating
random QBF instances, we give an overview of a general model for generating random DQBF instances.	
%
The tool is under development. 
%We are developing tools for improving correctness of tools like fuzzer (what is the random generator) and delta-debugger.
%    
\end{abstract}



\section{Introduction}
\label{sec:Intro}
Dependency Quantified Boolean Formulas (DQBF) are an extension of QBF
which allows the specification of non-linear dependencies between quantified variables. 
%
DQBF is NEXPTIME-complete, compared to the PSPACE completeness of QBF.

As an example for a DQBF, consider the formula
\begin{eqnarray*}
	F & := & \fa x_1,x_2,x_3 \ex y_1(x_1,x_2) \ex y_2(x_2,x_3) \ex y_3(x_1): F_0\\
	F_0 & := & (y_1 \vee x_1) \wedge (\overline{y_1} \vee x_2) \wedge (\overline{y_2} \vee \overline{x_2} \vee x_3) 
	%\wedge (y_3 \vee %\overline{x_1} \vee x_2) \wedge (\overline{y_3} \vee x_1).
\end{eqnarray*}

XXX
\section{preprocessing}
\label{sec:pre}

\subsection{Autarkies}
\label{sec:aut}
An autarky for a CNF $F$ is a partial assignment which satisfies every clause of $F$ containing an assigned variable. 
%
Clauses satisfied by some autarky can be removed satisfiability-equivalently. 
%
This is generalised to DQBF by considering partial assignments to existential variables and allowing boolean functions as values (fulfilling the dependencies), where the clauses with assigned variables need now to become tautologies.
%

A DQBF is called \textbf{lean} if it has no non-trivial autarkies.
The union of two lean DQCNF with compatible variables and dependencies is again lean, and thus every DQBF has a largest lean sub-DQBF, the \textbf{lean kernel}.
We denote the lean kernel by $\bm{\na(F)}$ (``N'' like ``normal form'').
Alternatively one can arrive at the lean kernel via \textbf{autarky reduction}.
For an autarky $\vp$ of a DQBF we denote by $\vp * F$ the DQBF with the clauses removed which are satisfied by $\vp$.
\begin{lem}\label{lem:autsateq}
	$\vp * F$ is satisfiability-equivalent to $F$ for an autarky $\vp$ of $F$.
\end{lem}
For two autarkies $\vp, \psi$ of $F$ one can consider the composition $\vp \circ \psi$, which on the variables of $\psi$ acts like $\psi$, and otherwise like $\vp$:
\begin{lem}\label{lem:compaut}
	The composition of two autarkies is again an autarky.
\end{lem}
Now the lean kernel is obtained by repeatedly applying autarky-reduction on $F$ as long as possible:
\begin{lem}\label{lem:decomp}
	Consider a DQBF $F$. The largest lean sub-DQBF $\na(F)$ is also obtained by applying autarky-reduction to $F$ as long as possible (in any order).
\end{lem}

We present two types of autarky systems for DQBF, namely, $A$ and $E$ (using ``A" to denote universal variables, and ``E" for existential variables).

Consider a DQCNF $F$ and $k \ge 0$:
\begin{itemize}
	\item An \textbf{$\Aaut_k$-autarky} for $F$ is an autarky such that all boolean functions assigned depend essentially on at most $k$ variables.
	\item An \textbf{$\Eaut_k$-autarky} is an autarky assigns at most $k$ (existential) variables.
\end{itemize}

\subsubsection{A}
$A_k$ restrict the boolean functions to depend on $k$ universal variable.

\subsection{E}
$E_1$ only uses one existential variable.

\subsection{Symmetry}
\label{sec:sym}

\section{DQBF solver development}
\label{sec:dev}

\section{Conclusion}
\label{sec:conc}

We consider $\Eaut_2 + \Aaut_2$ only as a first example and appetiser --- all the SAT-theory on autarkies can be combined with many interesting classes of boolean functions, to yield interesting autarky systems.
The results on the normalforms by the basic autarky systems $\Aaut_0, \Aaut_1, \Eaut_1$ and their combinations, on all known QCNF and DQCNF instances (as in QBFLIB), will be made available online.
After establishing this precise basis, the use of autarkies in pre- and inprocessing is the main question.



\bibliographystyle{plainurl}

%\BibliographyOKlibrary



\end{document}

%%% Local Variables:
%%% mode: latex
%%% TeX-master: t
%%% End:
