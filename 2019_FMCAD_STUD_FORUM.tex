% Oliver Kullmann, 18.5.2019 (Swansea)
% Bibtex entry \cite{KullmannShukla2019DQCNF}.
% Underlying report: 2018_DQBF.tex \cite{KullmannShukla2018DQBF}.
% Submitted to FMCAD 2019 https://fmcad.forsyte.at/FMCAD19/ :
% ID 79d524970076dd8509ffbd92da4922bba7682e3b.

\documentclass[conference]{IEEEtran}
\IEEEoverridecommandlockouts


\input Latex_macros/Definitionen.tex

\usepackage{enumerate}
\usepackage[all]{xy}
\usepackage[active]{srcltx}

\interdisplaylinepenalty=2500

\DeclareMathOperator{\varess}{\var_{es}} % essential variables

\DeclareMathOperator{\Aaut}{A}
\DeclareMathOperator{\Eaut}{E}


\begin{document}

\title{Improving Reasoning on DQBF}


\author{
\IEEEauthorblockN{Ankit Shukla}
\IEEEauthorblockA{
Johannes Kepler University, Austria\\
Email: ankit.shukla@jku.at}
}

\maketitle

\begin{abstract}
In this paper we investigate two techniques, autarkies and symmetry to improve the reasoning with the Dependency Quantified Boolean Formulas (DQBF).
%
DQBF extend QBF with non-linear dependencies between the quantified variables.
%
Inspired by the fruitfulness of the established model for generating
random QBF instances, we give an overview of a general model for generating random DQBF instances.	
%
The tool is under development. 
%We are developing tools for improving correctness of tools like fuzzer (what is the random generator) and delta-debugger.
%    
\end{abstract}



\section{Introduction}
\label{sec:Intro}
Dependency Quantified Boolean Formulas (DQBF) are an extension of QBF
which allows the specification of non-linear dependencies between quantified variables. 
%
In the case of QBF the variable dependencies always follow a
linear order (each existential variable depends on all universal variables left of its position in the quantifier prefix).
%
This implies DQBF can offers more succinct descriptions than
QBF, provided that the two classes do not collapse.
%
Deciding DQBF is NEXPTIME-complete, compared to the PSPACE completeness of QBF.
%
Many practical problems are known to be NEXPTIME-complete, e.g. partial information non-cooperative games or probabilistic planning of finite length, or certain bit-vector logics making DQBF an idea formalism to express them.

For example \par\noindent
%\begin{align}
%\nonumber\\
% \nonumber
%	%
%\end{align}
%
\begin{equation}
\begin{split}
&F := \fa x_1,x_2,x_3 \ex y_1(x_1,x_2) \ex y_2(x_2,x_3) \ex y_3(x_1): F_0 \\
&F_0 := (y_1 \vee x_1) \wedge (\overline{y_1} \vee x_2) \wedge (\overline{y_2} \vee \overline{x_2} \vee x_3) \\
&\hspace{2.3cm}\wedge (y_3 \vee \overline{x_1} \vee x_2) \wedge (\overline{y_3} \vee x_1)
\end{split}
\label{eqn:ex}
\end{equation}

is a DQBF formula where explicit dependencies ($(x_1,x_2), (x_2,x_3), (x_1)$) of quantified existential variables ($y_1, y_2, y_3$ respectively) are specified. We call the explicit dependencies the \textit{dependency set} of the corresponding variable. 
%
%For example, the dependency set of the existential variable $y_1$ is  $\{x_1,x_2\}$. 
%$\{x_2,x_3\}, \{x_1\}$

\section{preprocessing}
\label{sec:pre}

We present two techniques autarkies and symmetry that can be basis for pre-processing and in-processing of DQBF solving process. 
%
We have recently proposed the autarky theory with simple autarky systems (namely $A_0$, $A_1$ and $E_1$) for DQBF. 
%
Here we give an overview of the autarky theory and outline the more stronger autarky systems $A_2$, $E_2$ and $E_2 + A_2$. 
%

Symmetries are well understood and successfully exploited in the case of SAT. 
%
Recently general framework of symmetry for QBF was proposed.
%
Here, we present an overview of our current work of extending the symmetries from QBF to DQBF.   

\subsection{Autarkies}
\label{sec:aut}
An autarky for a CNF $F$ is a partial assignment which satisfies every clause of $F$ containing an assigned variable. 
%
Clauses satisfied by some autarky can be removed satisfiability-equivalently. 
%
This is generalised to DQBF by considering partial assignments to existential variables and allowing boolean functions as values (fulfilling the dependencies), where the clauses with assigned variables need now to become tautologies.
%
Note that an empty partial assignment is an autarky for every $F$ i.e. never touching any clause (\textit{trivial autarky}) and a satisfying assignment for $F$ is also an autarky for $F$, touching every clause and satisfying it.

A DQBF is called \textbf{lean} if it has no non-trivial autarkies.
The union of two lean DQCNF with compatible variables and dependencies is again lean, and thus every DQBF has a largest lean sub-DQBF, the \textbf{lean kernel}.
%We denote the lean kernel by $\bm{\na(F)}$ (``N'' like ``normal form'').
Alternatively one can arrive at the lean kernel via \textbf{autarky reduction}.
For an autarky $\vp$ of a DQBF we denote by $\vp * F$ the DQBF with the clauses removed which are satisfied by $\vp$.
\begin{lem}\label{lem:autsateq}
	$\vp * F$ is satisfiability-equivalent to $F$ for an autarky $\vp$ of $F$.
\end{lem}
For two autarkies $\vp, \psi$ of $F$ one can consider the composition $\vp \circ \psi$, which on the variables of $\psi$ acts like $\psi$, and otherwise like $\vp$:
\begin{lem}\label{lem:compaut}
	The composition of two autarkies is again an autarky.
\end{lem}
Now the lean kernel is obtained by repeatedly applying autarky-reduction on $F$ as long as possible:
\begin{lem}\label{lem:decomp}
	Consider a DQBF $F$. The largest lean sub-DQBF is also obtained by applying autarky-reduction to $F$ as long as possible (in any order).
\end{lem}

We present two types of autarky systems for DQBF, namely, $A$ and $E$ (using ``A" to denote universal variables, and ``E" for existential variables).
%
Consider a DQCNF $F$ and $k \ge 0$:
\begin{itemize}
	\item An \textbf{$\Aaut_k$-autarky} for $F$ is an autarky such that all boolean functions assigned depend essentially on at most $k$ variables.
	\item An \textbf{$\Eaut_k$-autarky} is an autarky assigns at most $k$ (existential) variables.
\end{itemize}

%\subsubsection{A}
$A_0, A_1$ allow the boolean functions to essentially depend on 0 resp. 1 universal variable, while $E_1$ only uses one existential variable (for a single autarky).
%$A_0, A_1, A_2$ restrict the structure of the boolean function, by forcing the existential variable to depend on at most 0, 1, 2 universal variable.
%\subsubsection{E}
%$E_1, E_2$ only uses one and two existential variable.
%

Consider the Example~\ref{eqn:ex}
Since $\overline{y_2}$ is pure, we have the $\Aaut_0$-autarky $y_2 \rightarrow 0$ (removing the third clause).
Furthermore we have the $\Aaut_1$-autarky $y_3 \rightarrow x_1$, removing the fourth and fifth clauses.
Both these autarkies are also $\Eaut_1$-autarkies.
We obtain the reduction result $\fa x_1,x_2,x_3 \ex y_1(x_1,x_2) : (y_1 \vee x_1) \wedge (\overline{y_1} \vee x_2)$, 
%which is equivalent to the QBF $\fa x_1,x_2 \ex y_1 : (y_1 \vee x_1) \wedge (\overline{y_1} \vee x_2)$, 
which doesn't allow any further autarky. 
%
This is lean kernel of the original formula.

The autarky system $A_0, A_1$ and $E_1$ are too weak for general DQBF solving. Out of total 334 instances of DQBF track of QBFEVAL'18 we only found 4 instances with non-trivial autakies. We are analyzing stronger autarky systems, $A_2$ (function is dependent on 2 other universal variables), $E_2$ (only uses two existential variables) and their combination. 
%
%The complexity of finding an $A_2$ autarky is at second level of polynomial hierarchy whereas for $E_2$ it is not known.

\subsection{Symmetry}
\label{sec:sym}

\section{DQBF solver development}
\label{sec:dev}

\section{Conclusion}
\label{sec:conc}

We consider $\Eaut_2 + \Aaut_2$ only as a first example and appetiser --- all the SAT-theory on autarkies can be combined with many interesting classes of boolean functions, to yield interesting autarky systems.
The results on the normalforms by the basic autarky systems $\Aaut_0, \Aaut_1, \Eaut_1$ and their combinations, on all known QCNF and DQCNF instances (as in QBFLIB), will be made available online.
After establishing this precise basis, the use of autarkies in pre- and inprocessing is the main question.



\bibliographystyle{plainurl}

%\BibliographyOKlibrary



\end{document}

%%% Local Variables:
%%% mode: latex
%%% TeX-master: t
%%% End:
